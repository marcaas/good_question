\section{660}
% 1.1
\begin{question}
    \begin{equation*}
        \lim\limits_{x \to \infty} x^2 (2^{\frac{1}{x}} - 2^{\frac{1}{x + 1}}) = 
    \end{equation*}
\end{question}
% 1.2
\begin{question}
    设$ a, b $ 为常数,且$ \lim\limits_{x \to \infty} \ab(\sqrt[3]{1 - x^6} - ax^2 - b) = 0 $. 则$ a = (\quad), b = (\quad) $ . 
\end{question}
% 1.3
\begin{question}
    $ f(x) = x^2 (x + 1)^2 (x + 2)^2 \cdots (x + n)^2 $ , 则$ f^{\prime \prime}(0) =  $ 
\end{question}
% 1.4
\begin{question}
    设$ y = y(x) $ 由参数方程  $ 
    \begin{cases}
         x = \frac 1 2 \ln(1 + t^2)      \\
         y = \arctan t  
    \end{cases} $ 
    确定, 则$ \frac{\d y}{\d x} = (\quad) , \frac{\d^2 y}{\d x^2} = (\quad) , y = y(x) $在任意点处的曲率$ k = (\quad) $  
\end{question}
% 1.5
\begin{question}
    设$ f(x) = \ln(\frac{1 - 2x}{1 + 3x}), n \ge 2 $ , 则$ f^{(n)} (0) = (\quad) $ 
\end{question}
% 1.6
\begin{question}
    设有界函数$ f(x) $ 在$ \ab(c, +\infty) $ 内可导, 且$ \lim\limits_{x \to +\infty} f\prime(x) = b $ , 则$ b = (\quad) $ 
\end{question}
% 1.7
\begin{question}
    \begin{equation*}
        I = \int \frac{ \sqrt{x + 1} + 2 }{\ab(x + 1)^2 - \sqrt{x + 1}} \d x.
    \end{equation*}
\end{question}
% 1.8
\begin{question}
    设$ \lim\limits_{x \to 0} \frac{ \sin 6x - \ab(\sin x) f(x) }{ x^3 } = 0 $ , 则$ \lim\limits_{x \to 0} \frac{6 - f(x)}{x^2} = (\quad) $ 
\end{question}
% 1.9
\begin{question}
    设$ f(x) $ 在$ [a, = +\infty) $ , 连续, 则" $ \exists x_n \in [a, +\infty) $, 有$ \lim\limits_{n \to \infty} x_n = +\infty $ 且$ \lim\limits_{n \to \infty} f(x_n) = \infty $ ", 是$ f(x) $ 在$ [a, +\infty) $ 无界的$ (\quad) $ 条件.
\end{question}
% 1.10
\begin{question}
    设$ f(x) $ 在$ [a, +\infty) $ 连续, 则 " $ \exists x_n \in[a, +\infty) $ , 有 $ \lim\limits_{n \to \infty} = +\infty $ 且$ \lim\limits_{n \to \infty} f(x_n) = \infty $ " 是$ f(x) $ 在$ [a, +\infty) $ 无界的(\quad)条件.
\end{question}
% 1.11
\begin{question}
    设函数$ f(x) $ 在$ (-\infty, +\infty) $ 上有定义,则下述命题中正确的是$ (\quad) $ 
    \begin{tasks}
        \task   若$ f(x) $ 在$ (-\infty, +\infty) $ 上可导且单调增加,则对一切$ x \in (-\infty, +\infty) $ 都有$ f'(x) > 0 $ 
        \task   若$ f(x) $ 在点$ x_0 $ 处取得极值, 则$ f'(x_0) > 0 $ 
        \task   若$ f''(x_0) = 0 $ , 则$ (x_0, f(x_0)) $ 是曲线$ y = f(x) $ 的拐点坐标
        \task   若$ f'(x_0) = 0 $ , $ f''(x_0) = 0 $ , $ f'''(x_0) \neq 0 $ , 则$ x_0 $ 一定不是$ f(x) $ 的极值点
    \end{tasks}
\end{question}
% 1.12
\begin{question}
    曲线$ 
    \begin{cases}
        x &= a(t - sint)    \\
        y &= a(1 - cost)
    \end{cases}
     $ , $ (a > 0) $ 在$ t = \frac{\pi}{2} $ 对应点处的曲率为$ (\quad) $ .
\end{question}
% 1.13
\begin{question}
    设$ f(x) $ 在$ (a, +\infty) $ 有界的$ (\quad) $ 条件.
\end{question}
% 1.14
\begin{question}
    设$ F(x) $ 是$ f(x) $ 在$ (a, b) $ 内的一个原函数,则$ f(x) + F(x) $ 在$ (a, b) $ 内$ (\quad) $ 
    \begin{tasks}
        \task   可导
        \task   连续
        \task   存在原函数
        \task   是初等函数
    \end{tasks}
\end{question}
% 1.15
\begin{question}
    判断下列说法是否正确, 若正确, 请证明; 若不正确, 请给出反例.
    \begin{tasks}
        \task   设$ \abs{f(x)} $ 在$ [a, b] $ 内可积, 则$ f(x) $ 在$ [a, b] $ 内可积.
        \task   设$ f^2(x) $ 在$ [a, b] $ 内可积, 则$ f(x) $ 在$ [a, b] $ 内可积.
    \end{tasks}
\end{question}
% 1.16
\begin{question}
    设常数$ \alpha > 0 $ , $ I_1 = \int_{0}^{\frac{\pi}{2}} \frac{\cos x}{1 + x^{\alpha}} \d x $ , $ I_2 = \int_{0}^{\frac{\pi}{2}} \d x $ , 则$ I_1 $ 与$ I_2 $ 的大小关系为?
\end{question}
% 1.17
\begin{question}
    设$ f(x) $ 为连续函数, $ \int_{0}^{\frac{\pi}{2}} f(x \cos x) \cos x \d x = A $ , 则$ \int_{0}^{\frac{\pi}{2}} f(x \cos x) x \sin x \d x = (\quad) $ 
    \begin{tasks}(4)
        \task   $ A $ 
        \task   $ -A $ 
        \task   $ 2A $ 
        \task   $ 0 $ 
    \end{tasks}
\end{question}
% 1.18
\begin{question}
    设$ f(x) = \begin{cases}
        x^2 &, x \ge 0   \\
        \cos x &, x < 0
    \end{cases} $ , $ g(x) = \begin{cases}
        x \sin \frac 1 x &, x \neq 0  \\
        0 &, x = 0
    \end{cases} $ , 则在区间$ (-1, 1) $ 内, 二者是否存在原函数?
\end{question}
