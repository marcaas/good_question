\section{660}
% 1.1
\begin{question}
    \begin{equation*}
        \lim\limits_{x \to \infty} x^2 (2^{\frac{1}{x}} - 2^{\frac{1}{x + 1}}) = 
    \end{equation*}
\end{question}
% 1.2
\begin{question}
    设$ a, b $ 为常数,且$ \lim\limits_{x \to \infty} \ab(\sqrt[3]{1 - x^6} - ax^2 - b) = 0 $. 则$ a = (\quad), b = (\quad) $ . 
\end{question}
% 1.3
\begin{question}
    $ f(x) = x^2 (x + 1)^2 (x + 2)^2 \cdots (x + n)^2 $ , 则$ f^{\prime \prime}(0) =  $ 
\end{question}
% 1.4
\begin{question}
    设$ y = y(x) $ 由参数方程  $ 
    \begin{cases}
         x = \frac 1 2 \ln(1 + t^2)      \\
         y = \arctan t  
    \end{cases} $ 
    确定, 则$ \frac{\d y}{\d x} = (\quad) , \frac{\d^2 y}{\d x^2} = (\quad) , y = y(x) $在任意点处的曲率$ k = (\quad) $  
\end{question}
% 1.5
\begin{question}
    设$ f(x) = \ln(\frac{1 - 2x}{1 + 3x}), n \ge 2 $ , 则$ f^{(n)} (0) = (\quad) $ 
\end{question}
% 1.6
\begin{question}
    设有界函数$ f(x) $ 在$ \ab(c, +\infty) $ 内可导, 且$ \lim\limits_{x \to +\infty} f\prime(x) = b $ , 则$ b = (\quad) $ 
\end{question}
% 1.7
\begin{question}
    \begin{equation*}
        I = \int \frac{ \sqrt{x + 1} + 2 }{\ab(x + 1)^2 - \sqrt{x + 1}} \d x.
    \end{equation*}
\end{question}
% 1.8
\begin{question}
    设$ \lim\limits_{x \to 0} \frac{ \sin 6x - \ab(\sin x) f(x) }{ x^3 } = 0 $ , 则$ \lim\limits_{x \to 0} \frac{6 - f(x)}{x^2} = (\quad) $ 
\end{question}
% 1.9
\begin{question}
    设$ f(x) $ 在$ [a, = +\infty) $ , 连续, 则" $ \exists x_n \in [a, +\infty) $, 有$ \lim\limits_{n \to \infty} x_n = +\infty $ 且$ \lim\limits_{n \to \infty} f(x_n) = \infty $ ", 是$ f(x) $ 在$ [a, +\infty) $ 无界的$ (\quad) $ 条件.
\end{question}
% 1.10
\begin{question}
    设$ f(x) $ 在$ [a, +\infty) $ 连续, 则 " $ \exists x_n \in[a, +\infty) $ , 有 $ \lim\limits_{n \to \infty} = +\infty $ 且$ \lim\limits_{n \to \infty} f(x_n) = \infty $ " 是$ f(x) $ 在$ [a, +\infty) $ 无界的(\quad)条件.
\end{question}
% 1.11
\begin{question}
    设函数$ f(x) $ 在$ (-\infty, +\infty) $ 上有定义,则下述命题中正确的是$ (\quad) $ 
    \begin{tasks}
        \task   若$ f(x) $ 在$ (-\infty, +\infty) $ 上可导且单调增加,则对一切$ x \in (-\infty, +\infty) $ 都有$ f'(x) > 0 $ 
        \task   若$ f(x) $ 在点$ x_0 $ 处取得极值, 则$ f'(x_0) > 0 $ 
        \task   若$ f''(x_0) = 0 $ , 则$ (x_0, f(x_0)) $ 是曲线$ y = f(x) $ 的拐点坐标
        \task   若$ f'(x_0) = 0 $ , $ f''(x_0) = 0 $ , $ f'''(x_0) \neq 0 $ , 则$ x_0 $ 一定不是$ f(x) $ 的极值点
    \end{tasks}
\end{question}
% 1.12
\begin{question}
    曲线$ 
    \begin{cases}
        x &= a(t - sint)    \\
        y &= a(1 - cost)
    \end{cases}
     $ , $ (a > 0) $ 在$ t = \frac{\pi}{2} $ 对应点处的曲率为$ (\quad) $ .
\end{question}
% 1.13
\begin{question}
    设$ f(x) $ 在$ (a, +\infty) $ 有界的$ (\quad) $ 条件.
\end{question}
% 1.14
\begin{question}
    设$ F(x) $ 是$ f(x) $ 在$ (a, b) $ 内的一个原函数,则$ f(x) + F(x) $ 在$ (a, b) $ 内$ (\quad) $ 
    \begin{tasks}
        \task   可导
        \task   连续
        \task   存在原函数
        \task   是初等函数
    \end{tasks}
\end{question}
% 1.15
\begin{question}
    判断下列说法是否正确, 若正确, 请证明; 若不正确, 请给出反例.
    \begin{tasks}
        \task   设$ \abs{f(x)} $ 在$ [a, b] $ 内可积, 则$ f(x) $ 在$ [a, b] $ 内可积.
        \task   设$ f^2(x) $ 在$ [a, b] $ 内可积, 则$ f(x) $ 在$ [a, b] $ 内可积.
    \end{tasks}
\end{question}
% 1.16
\begin{question}
    设常数$ \alpha > 0 $ , $ I_1 = \int_{0}^{\frac{\pi}{2}} \frac{\cos x}{1 + x^{\alpha}} \d x $ , $ I_2 = \int_{0}^{\frac{\pi}{2}} \d x $ , 则$ I_1 $ 与$ I_2 $ 的大小关系为?
\end{question}
% 1.17
\begin{question}
    设$ f(x) $ 为连续函数, $ \int_{0}^{\frac{\pi}{2}} f(x \cos x) \cos x \d x = A $ , 则$ \int_{0}^{\frac{\pi}{2}} f(x \cos x) x \sin x \d x = (\quad) $ 
    \begin{tasks}(4)
        \task   $ A $ 
        \task   $ -A $ 
        \task   $ 2A $ 
        \task   $ 0 $ 
    \end{tasks}
\end{question}
% 1.18
\begin{question}
    设$ f(x) = \begin{cases}
        x^2 &, x \ge 0   \\
        \cos x &, x < 0
    \end{cases} $ , $ g(x) = \begin{cases}
        x \sin \frac 1 x &, x \neq 0  \\
        0 &, x = 0
    \end{cases} $ , 则在区间$ (-1, 1) $ 内, 二者是否存在原函数?
\end{question}

% 1.19
\begin{question}
    微分方程$ yy'' + 2(y')^2 = 0 $ 满足初始条件$ y(0) = 1, y'(0) = - 1$的特解是(\quad)
\end{question}

% 1.20
\begin{question}
    设$ y = y(x) $ 是二阶常系数线性微分方程$ y'' + 2my' + n^2 y = 0 $ 满足$ y(0) = a $ 与$ y'(0) = b $ 的特解,其中$ m > n > 0 $ ,则$ \int_{0}^{+\infty} y(x)\d x $ 
\end{question}

% 1.21
\begin{question}
    设$ u = u(\sqrt{x^2 + y^2})(r = \sqrt{x^2 + y^2} > 0) $ 有二阶连续的偏导数,且满足
    \begin{equation*}
        \frac{\p^2 u}{\p x^2} + \frac{\p^2 u}{\p y^2} - \frac{1}{x} \cdot \frac{\p u}{\p x} + u = x^2 + y^2
    \end{equation*}
    则$ u(\sqrt{x^2 + y^2}) = (quad) $ 
\end{question}

% 1.22
\begin{question}
    设连续函数$ z = f(x,y) $满足$ \lim\limits_{x \to 0, y \to 1} \frac{f(x \cdot y) - 2x + y - 2}{\sqrt{x^2 + (y - 1)^2}} = 0 $,则$ \d z|_{(0,1)} = (\quad) $   
\end{question}

% 1.23
\begin{question}
    t109

    交换积分次序$ \int_{-\frac{\pi}{4}}^{\frac{\pi}{2}} \d\theta \int_{0}^{2\cos\theta} f(r\cdot\cos\theta, r\cdot\sin\theta) r \d r = $ 
\end{question}

% 1.24
\begin{question}
    t114

    设$ D $ 为圆域$ x^2 + y^2 \le 2x + 2y $ ,则$ \iint_D xy\d x \d y = $ 
\end{question}

% 1.25
\begin{question}
    t116

    设$ D = \{(x,y) | -1 \le x \le 1, 0 \le y \le 2\} $ ,则$ I = \iint_D \sqrt{\abs{y - x^2}} \d x\d y = $ 
\end{question}

% 1.26
\begin{question}
    t117

    设$ f(x) $ 为连续函数,$ F(x) = \int_{1}^{x} \d v \int_{v}^{x} f(u) \d u (x > 1) $ ,则$ F'(x) =  $ 
\end{question}

% 1.27
\begin{question}
    t135

    设$ \lim\limits_{x \to 0} \frac{\sin 6x - (\sin x)\cdot f(x)}{x^3} = 0 $ ,则$ \lim\limits_{x \to 0} \frac{6 - f(x)}{x^2} = (\quad) $
    \begin{tasks}(4)
        \task   0   
        \task   35
        \task   36
        \task   \infty
    \end{tasks} 
\end{question}

% 1.28
\begin{question}
    t145

    设$ f(x) $ 在$ [a,+\infty) $ 连续,则“$ \exists x_n \in [a,+\infty) $ 有$ \lim\limits_{n \to \infty} x_n = +\infty $ 且$ \lim\limits_{n \to \infty} f(x_n) = \infty $ ”是$ f(x) $ 在$ [a,+\infty) $ 无界的$ (\quad) $ 
    \begin{tasks}(2)
        \task   充分必要条件
        \task   必要非充分条件
        \task   充要条件
        \task   既非充分又非必要条件
    \end{tasks}
\end{question}

% 1.29
\begin{question}
    t147

    设$ f(x) =\begin{cases}
        e^{\frac{1}{x^2 - 1}}, &|x| < 1 \\
        x^4 - bx^2 + c, &|x| \ge 1
    \end{cases} $ 可导,则$ (b, c) = $
    \begin{tasks}(4)
        \task (2, 1)
        \task (1, 0)
        \task $ ( \frac 1 2  , -\frac 1 2) $
        \task (3, 2)
    \end{tasks} 
\end{question}

% 1.30
\begin{question}
    t148

    设$ \lim\limits_{x \to x_0^+} f'(x) = \lim\limits_{x \to x_0^-} f'(x) = a $ 则
    \begin{tasks}
        \task $ f(x) $ 在$ x = x_0 $ 处必可导且$ f'(x_0) = a $ 
        \task $ f(x) $ 在$ x = x_0 $ 处必连续,但未必可导
        \task $ f(x) $ 在$ x = x_0 $ 处必有极限但未必连续
        \task 以上结论都不对
    \end{tasks}
\end{question}

% 1.31
\begin{question}
    t158

    设$ f(x) $ 在$ x_0 $ 可导,且$ f'(x_0) > 0 $ ,则$ \exists \delta > 0 $ ,使得
    \begin{tasks}
        \task   $ f(x) $ 在$ (x_0 - \delta, x_0 + \delta) $ 单调上升
        \task   $ f(x) > f(x_0), x \in (x_0 - \delta, x_0 + \delta), x \neq x_0 $ 
        \task   $ f(x) > f(x_0), x \in (x_0, x_0 + \delta) $ 
        \task   $ f(x) < f(x_0), x \in (x_0, x_0 + \delta) $
    \end{tasks}
\end{question}

% 1.32
\begin{question}
    t176

    设$ F(x) $ 是$ f(x) $ 在$ (a,b) $ 内的一个原函数,则$ f(x) + F(x) $ 在$ (a,b) $ 内
    \begin{tasks}
        \task   可导
        \task   连续
        \task   存在原函数
        \task   是初等函数
    \end{tasks}
\end{question}

% 1.33
\begin{question}
    t181

    判断:设$ f^2(x) $ 在$ [a,b] $ 可积,则$ f(x) $ 在$ [a,b] $ 可积
\end{question}

% 1.34
\begin{question}
    t185

    设常数$ a > 0, I_1 = \int_{0}^{\frac{\pi}{2}} \frac{\cos x}{1 + x^{\alpha}} \d x, I_2 = \int_{0}^{\frac{\pi}{2}} \frac{\sin x}{1 + x^{\alpha}} \d x $ ,则
    \begin{tasks}(2)
        \task   $ I_1 > I_2 $ 
        \task   $ I_2 > I_1 $ 
        \task   $ I_1 = I_2 $ 
        \task   大小关系与$ \alpha $ 有关
    \end{tasks}
\end{question}

% 1.35
\begin{question}
    t186

    下列用牛顿-莱布尼茨公式计算定积分的做法中,错误的有几个?
    \begin{tasks}
        \task   $ \int_{0}^{\pi} \sqrt{\sin^3 x - \sin^5 x} \d x = \int_{0}^{\pi} \sin^{\frac 3 2} x \cos x \d x = \frac 2 5 \sin^{\frac 5 2} x |_0^\pi = 0 $ 
        \task   $ \int_{-1}^{1} \frac{\d x}{x} = \ln |x||_{-1}^1 = 0 $ 
        \task   $ \int_{0}^{\pi} \frac{\sec^2 x}{2 + \tan^2 x} \d x = \frac{1}{\sqrt{2}} \arctan \frac{\tan x}{\sqrt{2}} |_0^{\pi} = 0 $ 
        \task   $ \int_{-1}^{1} \frac{\d}{\d x} (\arctan \frac{1}{x}) \d x = \arctan \frac{1}{x} |_{-1}^1 = \frac{\pi}{2} $ 
    \end{tasks}
\end{question}

% 1.36
\begin{question}
    t197

    函数$ F(x) = \int_{x}^{x + \pi} \ln(1 + \cos^2 t) \cos 2t \d t $ 
    \begin{tasks}(4)
        \task   为正数
        \task   为负数
        \task   恒为零
        \task   不是常数
    \end{tasks}
\end{question}

% 1.37
\begin{question}
    t198

    设$ f(x) $ 为连续函数,$ \int_{0}^{\frac{\pi}{2}} f(x\cos x) \cos x \d x = A $ ,则$ \int_{0}^{\frac{\pi}{2}} f(x\cos x)x \sin x \d x = $ 
    \begin{tasks}(4)
        \task   0
        \task   $ A $ 
        \task   $ -A $
        \task   $ 2A $  
    \end{tasks}
\end{question}

% 1.38
\begin{question}
    t199

    设$ f(x) = \begin{cases}
        x^2, &x \ge 0 \\
        \cos x , &x < 0
    \end{cases}, g(x) = \begin{cases}
        x\sin \frac 1 x , &x \neq 0 \\
        0, &x = 0
    \end{cases}$ ,则在区间$ (-1, 1) $ 内,二者是否存在原函数?
\end{question}

% 1.39
\begin{question}
    t202

    下列反常积分发散的是
    \begin{tasks}(4)
        \task   $ \int_{-1}^{1} \frac{1}{\sin x} \d x $ 
        \task   $ \int_{-1}^{1} \frac{1}{\sqrt{1 - x^2}} \d x $ 
        \task   $ \int_{0}^{+\infty} e^{-x^2} \d x $ 
        \task   $ \int_{2}^{+\infty} \frac{1}{x\ln^2 x} \d x $ 
    \end{tasks}
\end{question}

% 1.40
\begin{question}
    t204

    判断:
    \begin{tasks}
        \task   设$ f(x) $ 在$ (-\infty, +\infty) $ 连续是奇函数,则$ \int_{-\infty}^{+\infty} f(x) \d x = 0 $ 
        \task   设$ f(x) $ 在$ (-\infty, +\infty) $ 连续,又$ \lim\limits_{R \to +\infty} \int_{-R}^{R} f(x) \d x $ 存在,则$ \int_{-\infty}^{+\infty} f(x) \d x $ 收敛
        \task   若$ \int_{-\infty}^{0} f(x) \d x $ 与$ \int_{0}^{+\infty} f(x) \d x $ 均发散,则不能确定$ \int_{-\infty}^{+\infty} f(x) \d x $ 是否收敛  
    \end{tasks}
\end{question}
