\section{ZY}
\begin{question}
    \begin{equation}
        \nonumber
        \lim\limits_{ x \to 0 } \frac{ \int_{\sin x}^{x} \sqrt{ 3 + t^2 } \d t }{ x ( e^{x^2} - 1 ) } = 
    \end{equation}
\end{question}

\begin{question}
    \begin{equation}
        \nonumber
        \lim\limits_{ x \to 0 } \frac{1}{x} \ab( \cot x - \frac{1}{x} ) = 
    \end{equation}
\end{question}

\begin{question}
    求函数 $ f(x) = \lim\limits_{ n \to \infty } \frac{ x^{ n + 2 } - x^{ -n } }{ x^n + x^{ -n } } $ 的间断点,并判别间断点的类型。
\end{question}

\begin{question}
    设函数 $ f(x) = \lim\limits_{ n \to \infty } \frac{ x^2 + nx( 1 - x ) \sin ^2 \pi x }{ 1 - n \sin^2 \pi x } $,分析$ f(x) $的间断情况。 
\end{question}

\begin{question}
    设函数$ f(x) =\lim\limits_{ n \to \infty } \cos^n \frac{ 1 }{ n^x } \quad (0 < x < + \infty) $,则$ f(x) $在其间断点处的值为?
\end{question}

\begin{question}
    记$ f(x) = 27x^3 + 5x^2 - 2 $的反函数为$ f^{-1} $,求极限
    \begin{equation}
        \nonumber
        \lim\limits_{ x \to \infty } \frac{ f^{-1} (27x) - f^{-1} (x) }{ \sqrt[3]{x} }
    \end{equation}
\end{question}

\begin{question}
    \begin{equation}
        \nonumber
        \lim\limits_{x \to 0} \frac{ (1 + x) ^ \frac{2}{x} - e^2 [1 - \ln(1 + x)] }{x}
    \end{equation}
\end{question}

\begin{question}
    \begin{equation}
        \nonumber
        \lim\limits_{x \to 0} \frac{ 1 + \frac{1}{2} x^2 - \sqrt{ 1 + x^2 } }{ ( \cos x - e^{ \frac{x^2}{2} }) \sin \frac{x^2}{2} }
    \end{equation}
\end{question}

\begin{question}
    \begin{equation*}
        \lim\limits_{x \to 0} \frac{ ( 1 + x )^{ \frac{1}{x} } -( 1 + 2x )^{ \frac{1}{2x} } }{ \sin x }
    \end{equation*}
\end{question}

\begin{question}
    设函数$ f(x) = (1 + x)^{\frac{1}{x}} \quad (x > 0) $,证明:存在常数$ A, B $,使得当$ x \to 0^+ $时,恒有
    \begin{equation*}
        f(x) = e + Ax +Bx^2 + o(x^2)
    \end{equation*}   
    并求常数$ A, B $ .
\end{question}

\begin{question}
    设函数$ f(x) = \left\{
        \begin{aligned}
            &\frac{ \ln(1 + x^3) }{ \arcsin x - x }, & x < 0   \\
            &\frac{ e^{-1} + \frac{1}{2} x^2 + x -1 }{ x \sin \frac{x}{6} }, & x > 0
        \end{aligned}  
    \right. $ 
    $ g(x) = \frac{ e^{\frac{1}{x}} \arctan \frac{1}{x} }{ 1 + e^{\frac{2}{x}} } $ , 求$ \lim\limits_{x \to 0} f [ g( x ) ] $ 
\end{question}

\begin{question}
    设$ \alpha \ge 5 $ 且为常数, 则$ k $ 为何值时极限
    \begin{equation*}
        I = \lim\limits_{x \to + \infty} [( x^\alpha + 8x^4 +2 )^k - x]
    \end{equation*}
    存在, 并求此极限值.
\end{question}

\begin{question}
    求$ \lim\limits_{n \to \infty} [ \sqrt{n} ( \sqrt{ n + 1 } - \sqrt{n} ) + \frac{1}{2} ]^{\frac{ \sqrt{ n + 1 } + \sqrt{n} }{ \sqrt{ n + 1 } -\sqrt{n} }} $ 
\end{question}

\begin{question}
    设当$ a \le x \le b $时, $ a \le f(x) \le b $, 并设存在常数$ k, 0 \le k < 1 $, 对于$ [a, b] $上的任意两点$ x_1 $ 与$ x_2 $, 都有
    \begin{equation*}
        \abs{ f(x_1) - f(x_2) } \le k \abs{ x_1 - x_2 }
    \end{equation*}
    证明:
    \begin{enumerate}
        \item 存在唯一的$ \epsilon \in [a, b] $ 使$ f(\epsilon) = \epsilon $ ;
        \item 对于任意给定的$ x_1 \in [a, b] $ 定义$ x_{n + 1} = f(x_n), n = 1,2,\cdots $, 则$ \lim\limits_{n \to \infty} x_n $ 存在, 且$ \lim\limits_{n \to \infty} x_n = \epsilon $ . 
    \end{enumerate}      
\end{question}